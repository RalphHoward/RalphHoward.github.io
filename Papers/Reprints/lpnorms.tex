q\documentstyle{amsppt}
%\nologo
\magnification=\magstep 1
\def\T{$T:L^p[0,1]\to L^q[0,1]\ $}
\def\nfp{\| f\| _p}
\def\ntfp{\| Tf\| _p}
\def\ntfq{\| Tf\| _q}
\def\V{$V:L^p[0,1]\to L^q[0,1]\ $}
\def\ntpq{\| T\| _{p,q}}
\def\ntp{\| T\| _{p,p}}
\def\nvpq{\| V\| _{p,q}}
\def\nvp{\| V\|_{p,p}}

\widestnumber\key{HAALAAP}
\NoBlackBoxes

\topmatter
\title Norms of positive operators on $L^p$-spaces \endtitle
\author Ralph Howard*\\
        Anton R. Schep** \endauthor
\thanks{*Research supported in part by the National Science Foundation under
grant number DMS-8803585, **Research supported in part by a Research and
Productive Scholarship grant from the University of South Carolina}
\endthanks
\affil University of South Carolina \endaffil
\address {Department of Mathematics, University of South Carolina, Columbia,
SC 29208}
\endaddress
\subjclass{47A30, 47B38, 47G05}
\endsubjclass
\keywords{Operatornorms, Positive linear operator, Volterra operator}
\endkeywords
\abstract {Let $0\le T: L^p(Y,\nu) \to L^q(X,\mu)$ be a positive linear
operator and let $\ntpq$ denote its operator norm. In this paper a
method is given to compute $\ntpq$ exactly or to bound $\ntpq$ from above. As
an application the exact norm $\nvpq$ of the Volterra operator $Vf(x)= \int
_0^x f(t)dt$ is computed.} 
\endabstract

\endtopmatter

\document
\heading 1. Introduction\endheading
For $1\le p <\infty$ let $L^p[0,1]$ denote the Banach space of (equivalence
classes of) Lebesgue measurable functions on [0,1] with the usual norm $\nfp =
(\int_0^1 |f|^p dt)^{\frac 1p}$. For a pair $p,q$ with $1\le p,q<\infty$ and a
continuous linear operator \T the operator norm is defined as usual by
$$\ntpq = \sup \{ \ntfq :\nfp =1 \} . \tag 1-1$$
Define the Volterra operator \V by
$$Vf(x)= \int _0^x f(t)dt. \tag 1-2$$
The purpose of this note is to show that for a class of linear operators $T$
between $L^p$ spaces which are positive (i.e. $f \ge 0$ a.e. implies $Tf\ge 0
$ a.e.) the problem of computing the exact value of $\ntpq$ can be reduced to
showing that a certain nonlinear functional equation has a nonnegative
solution. We shall illustrate this by computing the value of $\nvpq$ for $V$
defined by (1-2) above.

We first state this result. If $1<p<\infty$ then let $p'$ denote the conjugate
exponent of $p$, i.e. $p' =\frac p{p-1} $ so that $\frac1p +\frac 1{p'} =1$.
For $\alpha,\beta >0$ let $$B(\alpha ,\beta )= \int_0^1 t^{\alpha -1}
(1-t)^{\beta -1} dt$$ be the Beta function.

\proclaim{Theorem 1}
If $1<p,q<\infty$ then the norm $\nvpq$ of the Volterra operator \V is
$$\nvpq = (p')^{\frac 1q} q^{\frac 1{p'}} (p+q')^{\frac {q-p}{pq} } B(\frac
1q, \frac 1{p'})^{-1} \tag 1-3$$ In the case $p=q$ this
reduces to $$\nvp =\frac {p^ \frac 1{p'}(p')^\frac 1p}{B(\frac 1p ,\frac
1{p'})}\tag 1-4$$
\endproclaim
 
Special cases of this theorem are known. When $p=q=2k$ is an even integer, then
the result is equivalent to the differential inequality of section $7.6$ of
\cite {H-L-P}. This seems to be the only case stated in the literature.
The cases that $p$ or $q$ equals $1$ or  $\infty$ are elementary. It is easy to
see that $\|V\|_{p,\infty } =\| V\|_{1,q} =1$ for $1\le p\le \infty$ and $1\le
q\le \infty $. It is also straightforward for $1<p\le \infty$ and $1\le q <
\infty$ that $\|V\|_{p,1}=(\frac 1{p'+1})^{\frac 1{p'}} $ and $\| V\|_{\infty
,q}= (\frac 1{q+1})^\frac 1q$. 

The proof of theorem 1  is based on a general result about compact positive
operators between $L^p$ spaces. This theorem in turn will be deduced from a
general result about norm attaining linear operators between smooth Banach
spaces (see section 2 for the exact statement of the result).

In what follows $(X,\mu )$ and $(Y,\nu )$ will be $\sigma$-finite measure
spaces. If $T: L^{p}(Y,\nu ) \to L^{q}(X,\mu )$ is a continuous linear
operator we denote by $T^*$ the adjoint operator $T^*: L^{q'}(X,\mu ) \to
L^{p'}(Y,\nu )$. For any real number $x$ let $\operatorname{sgn} (x)$ be the
sign of $x$ (i.e. $\operatorname{sgn} (x)=1 $ for $x>0$, $=-1$ for $x<0$ and
$=0$
for $x=0$ ). Then for any bounded linear operator $T: L^{p}(Y,\nu ) \to
L^{q}(X,\mu )$ with $1<p,q<\infty$ we call a function $0\neq f \in L^p(X,\mu)$
a {\it critical point\/} of $T$ if for some real number $\lambda$ we have
$$T^*(\operatorname{sgn} (Tf) |Tf|^{q-1}) =\lambda \operatorname{sgn}
(f)|f|^{p-1} \tag1-5$$ (such  function $f$ is at least formally a solution to
the Euler-Lagrange equation for the variational problem implicit in the
definition of $\ntpq$). In the case that $T$ is positive and $f\ge 0$ a.e.
(1-5) takes on the simpler form 
$$T^*((Tf)^{q-1})=\lambda f^{p-1}\tag 1-6$$
For future reference we remark that the value of $\lambda$ in (1-5) and (1-6)
is not invariant under rescaling of $f$. If $f$ is replaced by $cf$ for some
$c>0$ then $\lambda$ is rescaled to $c^{q-p}\lambda$. Recall that a bounded
linear operator $T:X\to Y$ between Banach spaces is called {\it norm
attaining\/} if for some $0\neq f \in X$ we have $\| Tf\|_Y= \|T\| \| f\|_X$.
In this case $T$ is said to attain its norm at $f$. The following theorem will
be proved in section 2.

\proclaim{Theorem 2} Let $1<p,q<\infty$ and let $T: L^{p}(Y,\nu ) \to
 L^{q}(X,\mu )$ be a bounded operator.
\roster
\item"(A)" If $T$ attains its norm at $f\in L^p(X,\mu)$, then $f$ is a
critical
point of $T$ (and so satisfies (1-5) for some real $\lambda$).
\item"(B)" If $T$ is positive and compact, then (1-6) has nonzero solutions. If
also any two nonnegative critical points $f_1,f_2$ of $T$ differ by a positive
multiple, then the norm $\ntpq$ is given by
$$\ntpq = \lambda^{\frac 1q} \nfp ^{\frac {p-q}q} \tag 1-7$$
where $f\neq 0$ is any nonnegative solution to (1-6)
\endroster
\endproclaim

In section 2 we give an extension of theorem 2(A) to norm attaining operators
between Banach spaces with smooth unit spheres and use this result to prove
theorem 2B.  Theorem 2 is closely related to results of Gra\'slewicz \cite
{Gr}, who shows that if $T$ is positive, $p\ge q$ and (1-6) has a solution
$f>0$ a.e. for $\lambda =1$, then $\ntpq =1$. In  section 4 of this paper we
indicate an extension of this result. We prove that if there exists a $0<f$
a.e. such that 
$$T^*(Tf)^{q-1}\le \lambda f^{p-1}, \tag 1-8
$$ then $\ntp \le \lambda^{\frac 1p}$ in case $p=q$ and in case $q<p$ we have
$\ntpq \le \lambda ^{\frac 1p} \| Tf\|_{q}^{1-\frac qp}$ under the additional
hypothesis that $Tf\in L^q$. Inequality (1-8) can be used to prove a classical
inequality of Hardy. Another application of this result is a factorization
theorem of Maurey about positive linear operators from $L^p$ into $L^q$. 

It is worthwhile remarking that in case $p=q=2$ the equation (1-5) reduces to
the linear equation $T^*Tf= \lambda f$. In this case theorem 2 is closely
related to the fact that in a Hilbert space the norm of a compact operator is
the square root of the largest eigenvalue of $T^*T$.


\heading 2. Norm attaining linear operators between smooth Banach spaces.
\endheading Let $E$ be a Banach space and let $E^*$ denote its dual space. If
$f^*\in E^*$ then we denote by $f^*(f)= <f,f^*>$   the value of $f^*$ at
$f\in E$. If $0\neq f\in E$ then $f^*\in E^*$ {\it norms\/} $f$ if $\| f^*\|
=1$ and $<f,f^*>=\| f\|$. By the Hahn-Banach theorem there always exist such
norming linear functionals. A Banach space $E$ is called {\it smooth \/} if for
every $0\neq f\in E$ there exists a unique $f^*\in E^*$ which norms $f$.
Geometrically this is equivalent with the statement that at each point $f $ of
the unit sphere of $E$ there is a unique supporting hyperplane. It is well
known that $E$ is smooth if and only if the norm is G\^ateaux differentiable at
all points $0\neq f\in E$ (see e.g. \cite{B}). If $E$ is a smooth Banach space
and $0\neq f\in E$, then denote by $\Theta _E(f)$ the unique element of $E^*$
that norms $f$, note $\| \Theta_E (f)\| =1$. For the basic properties of smooth
Banach spaces and the continuity properties of the map $f \mapsto \Theta_E(f)$
we refer to \cite{B, part 3 Chapter 1}.

The basic examples of smooth Banach spaces are the spaces $L^p(X,\mu)$ where
$1<p<\infty$. For $0\neq f\in L^p(X,\mu)$ one can easily show that
$$\Theta_{L^p}(f)=\nfp^{-(p-1)}\operatorname{sgn} (f)|f|^{p-1}\tag 2-1$$
by considering when equality holds in H\"older's inequality. 

The following proposition generalizes part (A) of theorem 2 to norm attaining
operators between smooth Banach spaces.
\proclaim{Proposition} Let $T:E\to F$ be a bounded linear operator between
smooth
Banach spaces. If $T$ attains its norm at $0\neq f\in E$ then there exists a
real number $\alpha$ such that 
$$T^*(\Theta_F(Tf))=\alpha \Theta_E(f)\tag 2-2$$
and the norm of $T$ is given by 
$$\| T\| =\alpha \tag 2-3$$
\endproclaim
\demo{Proof} Define $\Lambda_1 ,\Lambda_2 \in E^*$ by
$$\align
\Lambda_1 (h) &=<h,\Theta_E (f)>\\
\Lambda_2 (h) &=\frac 1{\| T\|} <Th,\Theta_F(Tf)>=\frac 1{\|T\|}<h,
T^*(\Theta_F(Tf))>. 
\endalign$$
Then $\| \Lambda_1\| =1$ (since $\| \Theta_E(f)\| =1$) and $\Lambda_1(f)=\|
f\|$, so $\Lambda_1$ norms $f$. Similarly $\| \Theta_F(Tf)\| =1$ implies that
$\| \Lambda_2\| \le 1$, but using $\| Tf\| =\| T\| \| f\|$ we have
$\Lambda_2 (f)=\| f\|$. Therefore $\Lambda_2$ also norms $f$. The smoothness of
$E$ now implies that $\Lambda_1 =\Lambda_2$. Hence (2-2) holds with $\alpha
=\| T\|$ as claimed.
\enddemo

Theorem 2(A) now follows from the following lemma.
\proclaim{Lemma} If $E=L^p(X,\mu), F=L^q(Y,\nu)$ with $1<p,q<\infty$ and $f$ is
a solution of (2-2), then $f$ is a critical point of $f$, i.e.
$$T^*(\operatorname{sgn} (Tf) |Tf|^{q-1}) =\lambda \operatorname{sgn}
(f)|f|^{p-1} $$
where
$$\lambda = \alpha^q \nfp^{q-p} \tag 2-4$$
\endproclaim
\demo{Proof} First we note that if $f$ satisfies (2-2), then we have 
$$\ntfq = <Tf, \Theta_F(Tf)> =<f,T^*\Theta_F (Tf)=<f,\alpha \Theta_E
(f)>=\alpha \nfp .$$
Substitution of (2-1) into (2-2) and multiplication by $\ntfq ^{q-1}$ gives
$$T^*(\operatorname{sgn} (Tf)|Tf|^{p-1})=\alpha \ntfq^{q-1} \nfp^{-(p-1)}
\operatorname{sgn} (f)|f|^{p-1}=\alpha ^{q}\nfp^{q-p} \operatorname{sgn}
(f)|f|^{p-1}.$$
This completes the proof of the lemma and of theorem 2(A).
\enddemo

To prove theorem 2(B), we first make the observation that if $T:E\to F$ is a
compact linear operator and $E$ is reflexive, then $T$ attains its norm (since
every bounded sequence in $E$ contains a weakly convergent subsequence and $T$
maps weakly convergent sequences onto norm convergent sequences). If now $T$ is
a positive compact operator from $L^p(X,\mu)$ into $L^q(Y,\nu)$, then $T$
attains its norm at a nonnegative $f\in L^p(X,\mu)$ (simply replace  $f$
by $|f|$, if $T$ attains its norm at $f$). If  the additional hypothesis of
theorem 2(B) holds, then any other nonnegative critical point $f_0$ is a
positive multiple of $f$ and therefore $T$ also attains its norm at $f_0$.Now
the proposition and the lemma imply that $\| T\| =\alpha$, where $\alpha$
satisfies (2-4). Hence (1-7) holds. This completes the proof of Theorem 2.

\heading 3. The norm of the Volterra operator \endheading

In this section we shall prove Theorem 1. We first notice that the adjoint
operator of the Volterra operator is given by
$$V^*g(x)=\int_x^1 g(t)dt \text{ a.e.} \tag 3-1$$ Since $\int _0^x f(t)dt$ and
$\int _x^1 g(t)dt)$ are absolutely continuous functions, we can assume that
$V(f)$,respectively $V^*(g)$, equal these integrals everywhere. From Theorem
2(B) and the rescaling property of $\lambda$ it follows that to prove Theorem 1
it suffices to show that
$$V^*((Vf)^{q-1})=\lambda f^{p-1}\tag 3-2$$
has a unique positive solution in $L^p[0,1]$ normalized so that
$$Vf(1)=\int _0^1 f(t)dt=1\tag 3-3$$
Since $Vf$ is chosen to be absolutely continuous, we see that $V^*((Vf)^{q-1})$
can be chosen to be continuously differentiable on $[0,1]$. Hence any
nonnegative solution of (3-2) can be assumed to be continuously differentiable
on $[0,1]$. Also if $f$ is a nonnegative solution of (3-2) normalized so that
(3-3) holds, then $Vf$ is nonnegative and $Vf(1)=1$ so that $Vf$ is positive on
a neighborhood of $x=1$. From (3-1) we conclude that $V^*((Vf)^{q-1})$ is
positive on $[0,1)$. Hence any nonnegative solution of (3-1) and (3-2) can be
assumed to be strictly positive  and continuously differentiable on
$[0,1)$. Assume now that $f$ is such a solution of (3-1) satisfying (3-2). Take
the derivative on both sides in (3-1) and then multiply both sides by $f$
to get the following differential equation
$$-(Vf)^{q-1}f=\lambda (p-1)f^{p-1}f'.\tag 3-4$$
Using that $f$ is the derivative of $Vf$, we can integrate both sides to get
$$\frac 1q-\frac 1q (Vf)^{q}=\frac {\lambda (p-1)}p f^p\tag 3-5$$
since $Vf(1)=1$ and $f(1)=0$ by (3-2). To simplify the notation we let
$v(x)=Vf(x)$. Then $v(x)>0$ for $x>0$, $v'(x)>0$ for $x<1$, $v'(1)=0$ and (3-5)
becomes
$$\frac 1q (1-v(x)^q)= \frac {\lambda (p-1)}{p} v'(x)^p \tag 3-6$$
or
$$c_{p,q}= \frac {v'(x)}{\root p \of{1-v(x)^q}} \tag 3-7$$
where
$$c_{p,q}= (\frac p{\lambda q(p-1)})^{\frac 1p}. \tag 3-8$$
Using that $v(0)=0$ we can integrate (3-7) to get 
$$c_{p,q} x =\int _0^{v(x)} \frac 1{\root p \of{ 1-t^q}} dt. \tag 3-9$$
Putting $x=1$ in this equation we get
$$c_{p,q}=\int_0^1 \frac 1{\root p \of{1-t^q}} dt=\frac 1q B(\frac 1q,1- \frac
1p) =\frac 1q B(\frac 1q, \frac 1{p'}). \tag 3-10$$ 
The integral in (3-10) was reduced to the Beta function by the change of
variable $t=u^{\frac 1q}$. The equations (3-9) and (3-10) uniquely determine
the function $v$ and therefore also $f=v'$ and the number $\lambda$. This shows
that  (3-2) and (3-3) have a unique nonnegative solution. Moreover starting
with $v$ and $\lambda$ given by (3-9) and (3-10) one sees by working backwards
that $f=v'$ is a nonnegative solution of (3-2) and (3-3). Therefore by Theorem\
2(B) the norm of $V$ is given by (1-7). 
>From equations (3-10) and (3-8) we can solve for $\lambda$ to
obtain $$\lambda^{\frac 1q} =\frac {(p')^{\frac 1q}q^{\frac {p-1}q}}{B(\frac
1q ,\frac 1{p'})^{\frac pq}}. \tag 3-11$$
In case $p=q$ this shows that $\nvp =\lambda^{\frac 1p}$, which proves (1-4).
In
case $p \ne q$ we need to compute $\| v'\|_p =\nfp $. To do this, multiply
(3-7) by $\root p \of{1-v^q}$, raise the result to the power
$p-1$, and then multiply by $v'$ to obtain
$$v'(x)^p = c_{p,q}^{p-1}v'(x) (1-v(x)^q)^{\frac {p-1}p}. \tag3-12$$
Using that $v(0)=0$ and $v(1)=1$ we can integrate (3-12) to obtain
$$\aligned \nfp ^p &=\| v'\|_p^p =c_{p,q}^{p-1} \int ^1_0 (1-t^q)^{\frac
1{p'}}dt\\
&=c_{p,q}^{p-1}\frac 1q B(\frac 1q ,\frac 1{p'} +1)\\
&=c_{p,q}^{p-1}\frac 1q \frac {\frac 1{p'}}{\frac 1q +\frac 1{p'}} B(\frac
1q ,\frac 1{p'})\\
&=\frac {B(\frac 1q ,\frac 1{p'})^p}{q^{p-1} (p+q')}. \endaligned \tag 3-13$$
(Here we used the identity $B(\alpha ,\beta +1)=\frac \beta{\alpha +\beta}
B(\alpha ,\beta  )$. ) Therefore 
$$\nfp^{\frac {p-q}q}= \frac {B(\frac 1q ,\frac 1{p'})^{\frac
{p-q}{q}}}{q^{\frac {(p-1)(p-q)}{pq}} (p+q')^{\frac {p-q}{pq}}} \tag 3-14$$
Using (3-11) and (3-14) in formula (1-7) now gives formula (1-3) and the proof
of theorem 1 is complete.

\heading 4. Bounds for norms of positive operators \endheading
In this section we shall consider a positive operator $T$ acting on a space of
(equivalence classes of) measurable functions and give a necessary and
sufficient condition for $T$ to define a bounded linear operator from
$L^p(Y,\nu )$ into $L^q(X,\mu )$ where $1< q\leq p <\infty$ and obtain a
bound for $\ntpq$, similar to (1-7). Let $L^0(X,\mu)$ denote the space of a.e.
finite measurable functions on $X$ and let $M(X,\mu)$ denote the space of
extended real valued measurable functions on $X$. For some applications it is
useful to assume that $T$ is not already defined on all of $L^p$. Therefore we
shall assume that $T$ is defined on an {\it ideal} $L$ of measurable functions,
i.e. a linear subspace of $L^0(Y,\nu )$ such that if $f\in L$ and $|g| \le |f|$
in $L^0$, then $g\in L$. By $L_+$ we denote the collection of nonnegative
functions in $L$. A positive linear operator $T:L\to L^0(X,\mu )$ is called
{\it order continuous} if $0\leq f_n \uparrow f$  a.e. and $f_n , f \in L$
imply that $Tf_n \uparrow Tf$ a.e.. We first prove that such operators have
``adjoints''.

 
\proclaim {Lemma} Let $L$ be an ideal of measurable functions on $(Y,\nu )$ and
let $T$ be a positive order continuous operator from $L$ into $L^0(X,\mu )$.
Then there exists an operator $T^t:L^0(X,\mu )_+ \to M(Y,\nu)_+$ such
that for all $f\in L_+$ and all $g\in L^0(X,\mu )_+$ we have
$$\int_X (Tf)g d\mu = \int_Y f (T^tg) d\nu . \tag 4-1$$
\endproclaim

\demo{Proof} Assume first that there exists a function $f_0>0$ a.e. in $L$. Let
$g\in L^0(X,\mu )_+$. Then we define $\phi :L_+ \to [0,\infty]$ by
$\phi (f)= \int (Tf)g d\mu$. Since $Tf_0 <\infty$ a.e. we can find $X_1
\subset X_2 \subset \dots \uparrow X$ such that for all $n\ge 1$ we have
$$\int _{X_n} (Tf_0)g d\mu < \infty .$$
Let $L_{f_0} =\{ h: |h| \le cf_0 \text{ for some constant } c\}$ and define
$\phi_n :L_{f_0} \to \Bbb R$ by
$$\phi_n (h)= \int _{X_n}(Th)gd\mu .$$
The order continuity of $T$ now implies (through an application of  the
Radon--Nikodym theorem) that there exists a function $g_n\in L^1(Y,f_0d\nu )$
such that for all $h\in L_{f_0}$ we have
$$\phi_n (h)=\int_Y hg_n d\nu ,$$
see e.g. \cite{Z,theorem 86.3} . Moreover we can assume that $g_1 \le g_2 \le
\dots$ a.e.. Let $g_0 = \operatorname{sup} g_n$. An application of the monotone
convergence theorem now gives
$$\int_X (Th)gd\mu = \int_Y hg_0 d\nu$$
for all $0\le h\in L_{f_0}$. The order continuity of $T$ and another
application of the monotone convergence theorem now give
$$\int_X (Tf)gd\mu = \int_Y fg_0 d\nu \tag 4-2$$
for all $0\le f\in L$. If we put $T^t g= g_0$, then (4-2) implies that (4-1)
holds in case $L$ contains a strictly positive $f_0$. In case no such $f_0$
exists in $L$, then we can find via Zorn's lemma a maximal disjoint system
$(f_n)$ in $L^+$ and apply the above argument to the restriction of $T$ to the
functions $f\in L$ with support in the support $Y_n$ of $f_n$. We obtain that
way functions $g_n$ with support in $Y_n$ so that for all such $f$ we have
$$\int_X (Tf)gd\mu  = \int_{Y_n}fg_nd\nu$$
Now define $T^tg= \operatorname{sup} g_n$ and one can easily verify that in
this case again (4-1) holds. This completes the proof of the lemma.
\enddemo


The above lemma allows us to define for any positive operator $T:L\to
L^0(X,\mu)$ an adjoint operator $T^*$. Let $N= \{ g\in L^0(X,\mu): T^t(|g|)\
\in L^0(Y,\nu )\}$ and define $T^*g= T^tg^+ -T^tg^-$ for $g\in N$. It is easy
to see that $T^*$ is positive linear operator from $N$ into $L^0(Y,\nu)$ such
that $$\int_X (Tf)gd\mu = \int_Y f(T^*g)d\nu \tag 4-3$$
holds for all $0\le f\in L$ and $0\le g\in N$. Observe that in case $T:L^p \to
L^q$ is a bounded linear operator and $1\le p,q<\infty$ then $T^*$ as defined
as  above is an extension of the Banach space adjoint. The above construction
is motivated by the following example.


\demo{Example} Let $T(x,y)\ge 0$ be $\mu \times \nu$-measurable function on
$X\times Y$. Let $L= \{ f\in L^0(Y,\nu) \text { such that } \int
T(x,y)|f(y)|d\nu <\infty \text { a.e.}\}$ and define $T$ as the integral
operator $Tf(x)= \int_Y T(x,y)f(y) d\nu (y)$
on $L$. Then one can check (using Tonelli's theorem) that $N= \{ g\in
L^0(X,\mu) \text{ such that } \int_Y T(x,y)|g(x)|d\mu <\infty \text { a.e.}\}$
and that the operator $T^*$ as defined above is the the integral operator 
$\int_X T(x,y)g(x)d\mu (x)$.
\enddemo

We now present a H\"older type inequality for positive linear operators. The
result is known in ergodic theory (see \cite{K}, Lemma 7.4). We include the
short proof.
\proclaim{Abstract H\"older inequality} Let $L$ be an ideal of measurable
functions on $(Y,\nu )$ and let $T$ be a positive operator from $L$ into 
$L^0(X,\mu)$. If  $1<p<\infty$ and $p'=\frac p{p-1}$, then we have
$$T(fg)\le T(f^p)^{\frac 1p} T(g^{p'})^{\frac 1{p'}} \tag 4-4$$
for all $0\le f,g$ with $fg\in L$, $f^p\in L$ and $g^{p'}\in L$.
\endproclaim
\demo{Proof} For any two positive real numbers $x$ and $y$ we have the
inequality $x^{\frac 1p}y^{\frac 1{p'}}\le \frac 1p x+\frac 1{p'} y$, so that
if $0\le f,g$ with $fg\in L$, $f^p\in L$ and $g^{p'}\in L$, then for any
$\alpha >0$
$$\aligned T(fg)&= T((\alpha f)(\frac 1\alpha )g)\\
&\le \frac 1p T((\alpha f)^{p})+\frac 1{p'} T((\frac 1\alpha g)^{p'})\\
&= \frac 1p  \alpha^p T(f^p) +\frac1{p'} \frac 1{\alpha^{p'}}
T(g^{p'})\endaligned \tag 4-5
$$
Now for each $x\in X$ such that $T(f^p)(x)\neq 0$ choose the number $\alpha$
so that $\alpha^p T(f^p)(x)= \frac 1{\alpha^{p'}} T(g^{p'})(x)$. Then (4-5)
reduces to (4-4).
\enddemo

\proclaim{Theorem 3} Let $L$ be an ideal of measurable functions on $(Y,\nu )$
and let $T$ be a positive order continuous linear operator from $L$ into
$L^0(X,\mu )$. Let $1<q\le p<\infty$ and assume there exists $f_0\in L$ with
$0<f_0$ a.e. and there exists $\lambda >0$ such that 
$$T^*(Tf_0)^{q-1} \le \lambda f_0^{p-1} \tag 4-6$$ 
and in case $q<p$ also 
$$Tf_0\in L^q(X,\mu).\tag 4-7$$ 
Then $T$ can be extended to a positive linear map from $L^p(Y,\nu)$ into
$L^q(X,\mu)$ with 
$$\ntpq \le \lambda^{\frac 1p} \| Tf_0\|_q ^{1-\frac qp} \tag 4-8$$ 
in case $q<p$ and in case $p=q$ 
$$\ntp \le \lambda ^{\frac 1p}. \tag 4-9$$ 
If also $f_0\in L^p(Y,\nu)$, then 
$$\ntpq \le \lambda^{\frac 1q} \| f_0 \|_p^{\frac {p-q}q}. \tag 4-10$$ 
\endproclaim 

\demo{Proof} Define the positive linear operator $S:L^p(Y,\nu )\to L^0(X,\mu )$
by $Sf= (Tf_0)^{\frac {q-p}p }\cdot Tf$, note that $S=T$ in case $p=q$. Then it
is straightforward to verify that $S^*(h)= T^*((Tf_0)^{\frac {q-p}p }\cdot h)$.
This implies that 
$$S^*(Sf_0)^{p-1} = S^*((Tf_0)^{\frac {q(p-1)}p})=T^*(Tf_0 )^{q-1} \le \lambda
f_0^{p-1},$$
i.e. $S$ satisfies (4-6) with $p=q$. 
Let $Y_n= \{y\in Y: \frac 1n \le f_0(y) \le n\}$. Then $L^\infty
(Y_n,\nu) \subset L$. Let $0\le u \in L^\infty (Y_n,\nu)$. Then we have 
$$\align 
\int (Su)^p d\mu &= \int S( uf_0^{-\frac 1{p'}}f_0^{\frac 1{p'}})^p d\mu \\ 
&\le \int S(u^p 
f_0^{-p+1}) (Sf_0)^\frac p{p'} d\mu \text{ (Abstract H\"older
inequality)  } \\ 
&= \int u^pf_0^{-p+1}  S^*(Sf_0)^{(p-1)}d\nu \\ 
&\le \int u^pf_0^{-p+1} \lambda f_0^{p-1} d\nu  \text{  by (4-6)}  \\ 
&= \lambda \| u\|_p^p .
\endalign $$ 
Hence 
$$\| Su \|_p \le \lambda ^{\frac 1p} \| u\|_p  \tag 4-11$$
for all $0\le u\in L^\infty (Y_n,d\nu)$. If  $0\le u\in L$, let $u_n
=\operatorname{min} (u,n)\chi_{_{Y_n}}$. Then $u_n \uparrow u$ a.e. and (4-11)
holds for each $u_n$.  The order continuity of $T$ and the monotone convergence
theorem imply that $\| S\|_{p,p}\le \lambda^{\frac 1p}$. Note that in case
$p=q$ this proves (4-9). In case $q<p$ define the multiplication operator $M$, 
by $Mh= (Tf_0)^{\frac {p-q}p}\cdot h$. Then (4-7) implies, by means of
H\"older's inequality with $r= \frac pq, r'=\frac p{p-q} $, that $\| M\|_{p,q}
\le \| Tf_0 \|^{1-\frac qp}$. The inequality (4-8) follows now from the
factorization $T=MS$. 
Inequality (4-10) follows from (4-8) by using the inequality $\| Tf_0 \|_q \le
\ntpq \| f_0\|_p$ and solving for $\ntpq$. This completes the proof of the
theorem. \enddemo 

The above theorem is an abstract version of what is called the {\it Schur test}
for boundedness of integral operators (see \cite{H-S} for the case $p=q=2$ and
see \cite{G} ,Theorem 1.I for the case $1<q\le p<\infty$ ).     

\proclaim{Corollary} Let $L$ be an ideal of measurable functions on $(Y,\nu )$
and let $T$ be a positive order continuous linear operator from $L$ into
$L^0(X,\mu )$. Let $1<q\le p<\infty$ and assume there exists $f_0\in
L^p(Y,\nu)$ with $0<f_0$ a.e. and there exists $\lambda >0$ such that 
$$T^*(Tf_0)^{q-1} = \lambda f_0^{p-1}.\tag 4-12$$
Then $T$ can be extended to a positive linear map from $L^p(Y,\nu)$ into
$L^q(X,\mu)$ with 
$$\ntpq = \lambda^{\frac 1p} \| Tf_0\|_q ^{1-\frac qp}= \lambda^{\frac 1q} \|
f_0 \|_p^{\frac {p-q}q} \tag 4-13$$ 
and $T$ attains  its norm at $f_0$.
\endproclaim

\demo{Proof} If we multiply both sides of (4-12) by $f_0$ and then integrate,
we get
$$\int_X (Tf_0)^q d\mu =\lambda \int_Y (f_0)^p d\nu. \tag 4-14$$ This implies
that $Tf_0\in L^q(X,\mu)$, so that by the above theorem the inequalities (4-8)
and (4-10) hold. Equality (4-14) shows that $\| Tf_0\| _q =\lambda^{\frac 1q}
\|f_0\| _p^{\frac pq }$, from which it follows that $\ntpq \ge \lambda^{\frac
1q} \| f_0\|_p^{\frac pq -1} $. Hence we have equality in (4-10). From this it
easily follows that (4-13) holds and that $\| Tf_0\|_q =\ntpq \| f_0 \|_p$.
\enddemo


\demo{Remark} In the above corollary one could hope that in case $p=q$ the
equation (4-12) without the hypothesis $f_0\in L^p$ still would imply that
$\ntp =\lambda^{\frac 1p}$. Theorem 3 still gives inequality (4-9), but this is
all what can be said as can be seen from the following example. Let
$X=Y=[0,\infty)$ with $\mu =\nu$ equal to the Lebesgue measure and define the
integral operator $T$ by $Tf(x)=\frac 1x \int_0^x f(t)dt$. An easy
computation shows that for $1<p<\infty$ the equality (4-12) holds for some
constant $\lambda =\lambda (\alpha)$, whenever $f_0(y)= y^\alpha$ for all
$-1<\alpha < 0$. One can verify that in this case $\alpha =-\frac 1p$ gives the
best upperbound for $\ntp$,  in which case $\lambda =(\frac p{p-1})^p$.
Inequality (4-9) is then the classical Hardy inequality.\enddemo

We now state a converse to the above theorem, which
is essentially due to \cite{G, Theorem 1.II}. For the sake of completeness we
supply a proof, which is a simplification of the proof given in \cite{G}.

\proclaim{Theorem 4} Let $0\le T:L^p(Y,\nu )\to L^q(X,\mu )$ be a positive
linear operator and assume $1<p,q<\infty$. Then for all $\lambda $ with
$\lambda^{\frac 1q } > \ntpq$ there exists $0<f_0$ a.e. in $L^p(Y,\nu )$ such
that 
$$T^*(Tf_0 )^{q-1} \le \lambda f_0^{p-1}. \tag 4-15$$
\endproclaim

\demo{Proof} We can assume that $\ntpq = 1$. Then we assume that $\lambda >1$.
Now define $S:L^p(Y, \nu)_+ \to L^p(Y,\nu )_+$ by means of
$$Sf= (T^*(Tf)^{q-1})^{\frac 1{p-1}}. $$ Then it is easy to verify that $\nfp
\le 1$ implies that $\| Sf \|_p \le 1$, also that $0\le f_1 \le f_2$ implies
that $Sf_1 \le Sf_2$ and that $0\le f_n \uparrow f$ a.e. in $L^p$ implies that
$Sf_n \uparrow Sf$ a.e.. Let now $0<f_1$ a.e. in $L^p(Y, \nu )$ such that $\|
f_1\|_p \le \frac {\lambda -1}\lambda$. For $n>1$ we define $f_n=f_1 +\frac
1\lambda Sf_{n-1}$. By induction we verify easily that $f_n \le f_{n+1}$ and
that $\|f_n \|_p \le 1$ for all $n$. This implies that there exists $f_0$ in
$L^p$ such that $f_n \uparrow f_0$ a.e. and $\| f_0\| _p \le 1$. Now $Sf_n
\uparrow Sf_0$ implies that $f_0=f_1+\frac 1\lambda Sf_0$. Hence $Sf_0 \le
\lambda f_0$, which is equivalent to (4-15) and $f_0 \ge f_1 > 0$ a.e., so that
$f_0 >0$ a.e. and the proof is complete. \enddemo

We present now an application of the previous two theorems. The result
is due to Maurey (\cite{M} ).

\proclaim{Corollary } Let $0\le T:L^p(Y,\nu) \to L^q(X,\mu)$ a positive linear 
operator and assume $1<q<p<\infty$. Then there exists $0<g$ a.e. in
$L^r(X,\mu)$
with $\frac 1r =\frac 1q -\frac 1p$ such that $\frac 1g \cdot T: L^p(Y,\nu)
\to L^p(X,\mu )$.
\endproclaim

\demo{Proof} From the above theorem it follows that there exists $f_0\in
L^p (Y,\nu )$ such that (4-6) and (4-7) hold. The factorization follows now
from the proof of Theorem 3.
\enddemo
We conclude with another application of Theorem 3. An ideal $L$ of measurable
functions is called a Banach function space if $L$ is Banach space such that
$|g| \le |f|$ in $L$ implies $\| g \| \le \| f\|$.

\proclaim{Theorem 5} Let $L$ be a Banach function space and assume that $T$ and
$T^*$ are positive linear operators from $L$ into $L$. Then $T$ defines a
bounded linear operator from $L^2$ into $L^2$.
\endproclaim

\demo{Proof} Let $S=T^*T$. Then $S$ is a positive operator from $L$ into $L$,
so $S$ is continuous (see \cite{Z} ). Let $\lambda > r(S)$, where $r(S)$
denotes the spectral radius of $S$. From the Neumann series of the resolvent
operator $R(\lambda ,S) =(\lambda -S)^{-1}$ one sees that for all $0< g \in
L$ we have $f_0=R(\lambda ,S)g \ge \frac 1{\lambda} g>0$ and $Sf_0 \le \lambda
f_0$, i.e. $T^*(Tf_0)\le \lambda f_0$ so (4-6) holds with $p=q=2$. The
conclusion follows now from theorem 3.
\enddemo

A result for integral operators similar to the above theorem was proved in
\cite{S} ,by completely different methods.


\demo{Remark} With some minor modifications of the proofs one can show that
Theorems 3 and 4 and their corollaries also hold in case $0<q\le 1$.
\enddemo

 
\demo{Acknowledgements} The problem which motivated this paper (finding
methods to compute and estimate norms of concrete operators like the Volterra
operator) came up in a conversation between one of the authors and Ken Yarnall.
We also would like to thank Stephen Dilworth for pointing out that an earlier
version of the proposition in section 2 had a superfluous assumption.
\enddemo

\Refs

\ref\key{\bf B} \by B. Beauzamy \book Introduction to Banach spaces and their
geometry
\publ North--Holland \yr 1982
\endref

\ref\key {\bf G} \by E. Gagliardo \paper On integral transformations with
positive kernel \jour Proc. A.M.S \yr 1965 \pages 429--434 \vol 16 
\endref

\ref \key {\bf Gr} \by R. Grza\'slewics \paper On isometric domains of positive
operators on $L^p$--spaces \jour Colloq. Math \yr 1987 \pages 251-- 261
\vol LII
\endref

\ref \key{\bf H--L--P} \by G.H. Hardy, J.E. Littlewood and G. P\'olya
\book Inequalities \publ Cambridge University press \yr 1959 
\endref

\ref\key {\bf H--S} \by P.R. Halmos and V.S. Sunder \book Bounded Integral
Operators on $L^2$ Spaces \publ Springer--Verlag \yr 1978 
\endref

\ref\key {\bf K} \by U. Krengel \book Ergodic Theorems \publ De Gruyter
\yr 1985
\endref

\ref\key{\bf M} \by B. Maurey \paper Th\'eor\`emes de factorisation pour les
op\'erateurs lin\'eaires \`a valeurs dans les espaces $L^p$
\jour Ast\'erisque \vol 11 \yr 1974
\endref

\ref \key{\bf S} \by V.S. Sunder \paper Absolutely bounded matrices \jour
Indiana Univ. J. \yr 1978 \pages 919--927 \vol 27
\endref

\ref\key{\bf Z} \by A.C. Zaanen \book Riesz spaces II \yr 1983 \publ North
Holland
\endref

\endRefs
\enddocument

