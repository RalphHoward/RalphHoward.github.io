%%%  This document is a LaTeX2e document 
%%%  It uses the AMS article style  

\documentclass[11pt]{amsart}
 
%%% Make more symbols available 
\usepackage{amsmath}
\usepackage{amssymb}
\usepackage{latexsym}



%%% More flexible enumerate environments.

\usepackage{enumerate}


%%% Some packages useful for graphics 
% \usepackage{bezier}
\usepackage{epic,eepic}	
%\usepackage{epsfig}            % For postscript
\usepackage{graphicx}
 
%%% Bold Italic font.  My convention is to use this font when defining
%%% terms. The usage is the same as \emph{}

\newcommand{\bi}[1]{{\bf\itshape #1\/}}  % Bold Italic (used for def'ing 
                                  % terms) 
 

\newcommand{\eq}[1]{~\eqref{#1}}

%%%
%%%  Sized delimiters.  Note \( and \) and \[ \] override
%%%  LaTeX's start and stop math mode
%%%

\renewcommand{\(}{\left(}
\renewcommand{\)}{\right)}
\renewcommand{\[}{\left[}
\renewcommand{\]}{\right]}

%%%  For Marginal Notes.  Useful during editing, especially with 
%%%  several authors involved.

\catcode`@=11 \@mparswitchfalse  %This puts the \mnote's on the right.

\newcounter{mnotecount}[section]
\renewcommand{\themnotecount}{\thesection.\arabic{mnotecount}}
\newcommand{\mnote}[1]
{\protect{\stepcounter{mnotecount}}$^{\mbox{\footnotesize  $%\!\!\!\!\!\!\,
      \bullet$\themnotecount}}$ \marginpar{\null\hskip-9pt\vskip-30pt\raggedright\tiny\em
      \themnotecount:\! #1} }
%\newcommand{\mnote}[1]{}	 %Deletes mnotes for a final copy


\newcommand{\ralph}{{\tiny \bf *RH*} } % For labeling a \mnote as 
                                        % being by Ralph Howard.
%\newcommand{\co}{{\tiny\bf *CA*} }  % For labeling a \mnote as 
                                        % being by Co Author.


%%%  Theorem type environments

% \newtheorem{theorem}{Theorem}  % I have used this for theorems in 
                 % the introduction for labeling independent of the
                 % rest of the document. 

%\swapnumbers  % Changes num Theorem to Theorem num.  Only effects the   
               % environments listed below it.

\newtheorem{thm}{Theorem}
\newtheorem{lemma}[thm]{Lemma}
\newtheorem{prop}[thm]{Proposition}
\newtheorem{cor}[thm]{Corollary}
\newtheorem{claim}[thm]{Claim}
 
\theoremstyle{definition}

\newtheorem{notation}[thm]{Notation}
\newtheorem{defn}[thm]{Definition}

\theoremstyle{remark}

\newtheorem{remark}[thm]{Remark}
\newtheorem{example}[thm]{Example}
%\newtheorem{prob}{{\bf Problem}}


%%%%%% My standard abbreviations for mathematical symbols.  Some 
%%%%%% have variants for compatibility with coauthors. 

%%% Standard algebraic objects:
\newcommand{\R}{{\mathbb R}}
\newcommand{\Z}{{\mathbb Z}}
\newcommand{\Q}{{\mathbb Q}}

%%% Derivatives:

\newcommand{\f}{\partial}    
%\newcommand{\p}[1]{{\frac{\partial}{\partial #1}}}  % d/d#1 
                                                % (the vector field)
%\renewcommand{\d}{\,\mathsf{d}}   % Sometimes used as the "d" in an 
                                  % integral.

%%% Geometric and topological objects:

\newcommand{\area}{\operatorname{Area}\nolimits}
\newcommand{\dist}{\operatorname{dist}}  
\newcommand{\length}{\operatorname{Length}\nolimits}
\newcommand{\vol}{\operatorname{Vol}\nolimits}
 
\newcommand{\un}{{\mathbf n}}       % Unit Normal.
\newcommand{\ut}{{\mathbf t}}       % Unit Tangent.
\newcommand{\ub}{{\mathbf b}}       % Unit Binormal.
\newcommand{\ii}{\text{\sl I\!I}}   % Second fundamental form

%%% Misc.

\newcommand{\cd}{,\dots,}      % Commas with Dots. i.e. ,...,
\newcommand{\e}{\varepsilon}   % epsilon that looks small.
\newcommand{\ol}{\overline}   % OverLine
\newcommand{\cn}{\colon}
\newcommand{\la}{\langle}
\newcommand{\ra}{\rangle}

\newcommand{\hint}{\emph{Hint:} }



\newcounter{ProbCount}

\newcommand{\prob}{\stepcounter{ProbCount}
{\noindent{\bf \arabic{ProbCount}.} }}

\newcounter{SubProbCount}[ProbCount]
%\stepcounter{subpr}
\newcommand{\sub}{\stepcounter{SubProbCount}(\alph{SubProbCount}) }

\newcommand{\an}{\emph{Answer:} }
\newcommand{\sol}{\smallskip \emph{Solution:  }}



%%% Title

\title[]{}

%%% Subject and keyword information:

%\subjclass[2000]{}
%\keywords{}


\begin{document}
 

\thispagestyle{empty}


\centerline{\Large\bf Mathematics 300 Homework, \today.}\bigskip


Here are some examples of showing existence results.

\prob Show that there exist integers $x$ and $y$ such that
$x^2+y^2=25$.

\prob Show that every even integer is the sum of two odd integers.

\prob Show that every even number between 20 and 30 inclusive is the sum
of two prime numbers.
\bigskip

Read Chapter 8 up to Section 8.4 (that is pages 131--139).  
On Page 145 do Problems 1, 3, 7, and 27.

\newpage

\begin{proof}[Solution to Problem 1]
Let $x=3$ and $y=4$, then $x^2+y^2=3^2+4^2=9+16=25$.  This shows the
existence of integers $x$ and $y$ with $x^2+y^2=25$.
\end{proof}


\begin{proof}[Solution to Problem 2]
Before starting the proof let us look at some examples:
\begin{align*}
2&= 1+1\\
4 & = 3+1\\
6 &= 5+1\\
8&= 7+1\\
10 &= 9+1
\end{align*}

Let us now do the proof:  Let $a$ be an even integer.  Then $a=2k$ for
some integer $k$  Now write
$$
a = 2k = (2k-1)+1 .
$$
This shows that $a$ is the sum of the odd numbers $2k-1$ and $1$.
\end{proof}


\begin{proof}[Solution to Problem 3]
We just exhibit each even number $n$ with $20\le n \le 30$ as a sum of 
two primes.
\begin{align*}
20 &=3+17\\
22 & = 3 + 19\\
24 &= 5+ 19\\
26 & = 3 + 23\\
28 & = 5 + 23\\
30 & = 7+23.
\end{align*}


One of the most famous unsolved problems in mathematics
is to show that every even number greater than $4$ is
the sum of two primes.  This was conjectured by the 
German mathematician Christian Goldbach in 1742.  In the 275 years since then
many people have worked on this, but it unproved. It is know that
even even number greater than $4$ but 
less than $4{,}000{,}000{,}000{,}000{,}000{,}000$ is the sum of two 
primes.  
\end{proof}






\end{document}

**********************************
 
 \includegraphics[scale=1]{graph.pdf}

 or

\begin{figure}[hb]
\includegraphics[scale=1]{graph.pdf}
\caption[]{}
\label{}
\end{figure}

parameters are    height  ,  width  ,   angle

**********************************



\begin{figure}[hb]
\centering
\input{graph1.eepic}
\caption[]{}
\label{}
\end{figure}




