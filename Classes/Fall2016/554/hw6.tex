%%%  This document is a LaTeX2e document 
%%%  It uses the AMS article style  

\documentclass[11pt]{amsart}
 
%%% Uncomment (or comment) any of the following if you wish to use 
%%% (or not use) the feature.

%%% Running Headers and footers
%%%\usepackage{fancyheadings}

%%% Multipart figures
%\usepackage{subfigure}

%%% Make more symbols available 
\usepackage{amsmath}
\usepackage{amssymb}
\usepackage{latexsym}

%%% Some packages useful for graphics 
% \usepackage{bezier}
% \usepackage{epic,eepic}	
%\usepackage{epsfig}            % For postscript

%%% While not a package, here is a reminder on the usage of 
%%% \includegraphics:
%\includegraphics[height= ,width= , angle=, scale=]{filename}

%%% Surround parts of graphics with box
%\usepackage{boxedminipage}
 
%%% If you want to generate a toc for each chapter (use with book)
%  \usepackage{minitoc}
 
%%% Bold Italic font.  My convention is to use this font when defining
%%% terms. The usage is the same as \emph{}

\newcommand{\bi}[1]{{\bf\itshape #1\/}}  % Bold Italic (used for def'ing 
                                  % terms) 
%%% Name and addresses:

\author{Ralph Howard}

%\address{Department of Mathematics,
%University of South Carolina,
%Columbia, S.C. 29208, USA}
%\email{howard\char'100math.sc.edu}
%\urladdr{www.math.sc.edu/$\sim$howard}

%%% Name(s) and addresses of coauthors:

% \author{}
% 
% \address{Department of Mathematics,
% University_Name,
% Address}
% \email{}
% \urladdr{}

%%% Current date (Edit to give another date):

\date{\today}

%%%%%%  Formatting:

%%% Formating references to equations:

\newcommand{\eq}[1]{~\eqref{#1}}

%%%
%%%  Sized delimiters.  Note \( and \) and \[ \] override
%%%  LaTeX's start and stop math mode
%%%

\renewcommand{\(}{\left(}
\renewcommand{\)}{\right)}
\renewcommand{\[}{\left[}
\renewcommand{\]}{\right]}

%%%  For Marginal Notes.  Useful during editing, especially with 
%%%  several authors involved.

\catcode`@=11 \@mparswitchfalse  %This puts the \mnote's on the right.

\newcounter{mnotecount}[section]
\renewcommand{\themnotecount}{\thesection.\arabic{mnotecount}}
\newcommand{\mnote}[1]
{\protect{\stepcounter{mnotecount}}$^{\mbox{\footnotesize  $%\!\!\!\!\!\!\,
      \bullet$\themnotecount}}$ \marginpar{\null\hskip-9pt\vskip-30pt\raggedright\tiny\em
      \themnotecount:\! #1} }
%\newcommand{\mnote}[1]{}	 %Deletes mnotes for a final copy


\newcommand{\ralph}{{\tiny \bf *RH*} } % For labeling a \mnote as 
                                        % being by Ralph Howard.
%\newcommand{\co}{{\tiny\bf *CA*} }  % For labeling a \mnote as 
                                        % being by Co Author.


%%%  Theorem type environments

% \newtheorem{theorem}{Theorem}  % I have used this for theorems in 
                 % the introduction for labeling independent of the
                 % rest of the document. 

%\swapnumbers  % Changes num Theorem to Theorem num.  Only effects the   
               % environments listed below it.

\newtheorem{thm}{Theorem}
\newtheorem{lemma}[thm]{Lemma}
\newtheorem{prop}[thm]{Proposition}
\newtheorem{cor}[thm]{Corollary}
\newtheorem{claim}[thm]{Claim}
 
\theoremstyle{definition}

\newtheorem{notation}[thm]{Notation}
\newtheorem{defn}[thm]{Definition}

\theoremstyle{remark}

\newtheorem{remark}[thm]{Remark}
\newtheorem*{example}{Example}
\newtheorem{prob}{{\bf Problem}}


%%%%%% My standard abbreviations for mathematical symbols.  Some 
%%%%%% have variants for compatibility with coauthors. 

%%% Standard algebraic objects:

\newcommand{\C}{{\mathbf C}}   % Complex numbers
% \newcommand{\C}{{\mathbb C}}   % Complex numbers, blackboard bold 
                                 % version
\newcommand{\Q}{{\mathbf Q}}   % rational numbers
%\newcommand{\Q}{{\mathbb Q}}   % rational numbers, blackboard bold 
                                % version.
\newcommand{\R}{{\mathbf R}}   % Real Numbers
%\newcommand{\R}{{\mathbb R}}   % Real Numbers, blackboard bold 
							    % version.
\newcommand{\Z}{{\mathbf Z}}   % Integers 
%\newcommand{\Z}{{\mathbb Z}}   % Integers, blackboard bold 
							    % version.

%%% Standard geometric objects:

\newcommand{\s}{{\mathbf S}}   % The sphere
%\newcommand{\s}{{\mathbb S}}   % The sphere, blackboard bold 
							    % version.
\newcommand{\Hy}{{\mathbf H}}   % the Hyperbolic space
%\newcommand{\Hy}{{\mathbb H}}   % the Hyperbolic space, blackboard 
                                 % bold version.
\newcommand{\nothing}{\varnothing} % The empty set.	

%%% Standard Lie groups:

\newcommand{\GL}{{\mathbf{GL}}} % General Linear group			
%\newcommand{\GL}{{\mathbb{GL}}} % General Linear group, blackboard 
                                 % bold version.
\renewcommand{\O}{{\mathbf O}}  % Orthogonal group
%\renewcommand{\O}{{\mathbb O}}  % Orthogonal group, blackboard 
                                 % bold version.
\newcommand{\SO}{{\mathbf{SO}}} % Special Orthogonal group							
%\newcommand{\SO}{{\mathbb{SO}}} % Special Orthogonal group,
                                 % blackboard bold version.

%%% Derivatives:

\newcommand{\f}{\partial}    
%\newcommand{\p}[1]{{\frac{\partial}{\partial #1}}}  % d/d#1 
                                                % (the vector field)
%\renewcommand{\d}{\,\mathsf{d}}   % Sometimes used as the "d" in an 
                                  % integral.

%%% Geometric and topological objects:

\newcommand{\area}{\operatorname{Area}\nolimits}
\newcommand{\dist}{\operatorname{dist}}  
\newcommand{\length}{\operatorname{Length}\nolimits}
\newcommand{\vol}{\operatorname{Vol}\nolimits}
 
\newcommand{\con}{\nabla}     % CONection on a manifold.     
\newcommand{\acon}{\overline{\nabla}} % Ambient CONection.

\newcommand{\leb}{{\mathcal L}}     %Lebesue measure.
\newcommand{\hau}{{\mathcal H}}     % Hausdorff measure.

\newcommand{\un}{{\mathbf n}}       % Unit Normal.
\newcommand{\ut}{{\mathbf t}}       % Unit Tangent.
\newcommand{\ub}{{\mathbf b}}       % Unit Binormal.
\newcommand{\ii}{\text{\sl I\!I}}   % Second fundamental form

%%% Misc.

\newcommand{\cd}{,\dots,}      % Commas with Dots. i.e. ,...,
\newcommand{\e}{\varepsilon}   % epsilon that looks small.
\newcommand{\image}{\operatorname{Image}}
\newcommand{\ol}{\overline}   % OverLine
\newcommand{\Span}{{\operatorname{Span}}}
\newcommand{\trace}{\operatorname{tr}}
\newcommand{\cn}{\colon}
\newcommand{\la}{\langle}
\newcommand{\ra}{\rangle}
\newcommand{\lla}{\left\langle}
\newcommand{\rra}{\right\rangle}
\renewcommand{\setminus}{\smallsetminus}
\newcommand{\hint}{{\emph{Hint:} }}
\newcommand{\ds}{\displaystyle}
\newcommand{\x}{{\mathbf x}}
\newcommand{\y}{{\mathbf y}}
\newcommand{\z}{{\mathbf z}}
\newcommand{\nn}{\nonumber}
%%% Title

\title[]{}

%%% Subject and keyword information:

%\subjclass[2000]{}
%\keywords{}


\begin{document}
 
 
 
 \centerline{\LARGE\bf Math 554 \hfill Homework }\bigskip
 
To review a bit of what we talked about in class today: Let $E$ be a
metric space and $S\subseteq E$.  Then if $\{V_i\}_{i\in I}$ is a
collection of subsets of $E$, then $\{V_i\}_{i\in I}$ is a \bi{cover} of
$S$ iff $S\subset \bigcup_{i\in I}V_i$.  The collection $\{V_i\}_{i\in
I}$ is an \bi{open cover} of $S$ iff it is a cover of $S$ and each
$V_i$ is open.

\begin{defn}
The subset $S$ of the metric space $E$ is \bi{compact} iff for every 
open cover $\{V_i\}_{i\in I}$ of $S$ there is a finite collection
$\{V_{i_1},V_{i_2}\cd V_{i_n}\}\subseteq \{V_i\}_{i\in I}$
with $S\subset V_{i_1}\cup V_{i_2}\cup\cdots \cup V_{i_n}$.\qed
\end{defn}

Put more briefly and eloquently:  The set $S$ is compact iff every
open cover of $S$ has a finite subcover. 


\begin{prob}
Show every finite subset of a metric space is compact.\qed
\end{prob}

We have seen the following:

\begin{prop}
If $E$ is a compact metric space, then any closed subset of $S$ is compact.\qed
\end{prop}


There is a partial converse

\begin{prop}
If $S$ is a compact subset of a metric space $E$, then $S$ is closed
in $E$.
\end{prop}

\begin{prob}
Prove this. \hint It is easiest to prove the contrapositive:  If $S$
is not closed, then $S$ is not compact.  So assume that $S$ is not
closed.  Then $S$ has a limit point $p$ with $p\notin S$.  For any
$r>0$ let $V_r= \mathcal C \ol{B}(p,r)$ (that is $V_r$ is the
compliment of the closed ball $\ol{B}(p,r)$).  Then show
$\{V_r\}_{r>0}$ is an open cover of $S$.  If 
$\{V_{r_1},V_{r_2}\cd V_{r_n}\}$ is a finite subset of $\{V_r\}_{r>0}$
then 
\begin{align*}
V_{r_1}\cup V_{r_2}\cup \cdots \cup V_{r_n}
&=\mathcal C \ol B(p,r_1)\cup \mathcal C \ol B(p,r_2)\cup \cdots \cup 
\mathcal C \ol B(p,r_n)\\
&= \mathcal C \Big(B(p,r_1)\cap B(p,r_2)\cap \cdots \cap B(p,r_n)
\Big)\\
&=\mathcal C \ol{B}(p,r_*)
\end{align*}
where 
$
r_*=\min\{ r_1,r_2\cd r_n\}.
$
Now use that $p$ is a limit point to show this finite uniion can not
cover $S$.\qed
\end{prob}

\begin{defn}
Let $E$ be a metric space and $S\subseteq E$.  Then $p\in E$ is a 
\bi{cluster point} of $S$ iff every open ball, $B(p,r)$, about $p$
contains infinitely points of~$S$.\qed
\end{defn}

Note that it is not required that the cluster point be in $S$.  For
example for the open interval $(a,b)$ the set of cluster points is the
closed interval $[a,b]$.  Of these the points $a$ and $b$ are not in $(a,b)$.

\begin{prob}
For the following sets, $S$, give the set of cluster points of the set and
say which of these are in $S$.
\begin{enumerate}
\item[(a)]  The open ball $S=B((0,0),r)$ of radius $r$ about the
  origin in $\R^2$.
\item[(b)] The set $S=\mathbf Q$ of rational numbers in $\R$.\qed
\end{enumerate}
\end{prob}

\begin{thm}
Show that every infinite subset of a compact metric space has a
cluster point.
\end{thm}

\begin{prob}
Prove this. \hint Let $E$ be a compact metric space and $S\subseteq E$
an infinite set.  Towards a contradiction assume that $S$ does not
have a cluster point.  Then show for for each $p\in E$ there is a $r_p>0$ such
that $B_{r_p}$ only contains a finite number of points of $S$.  Show
$\{B(p,r_p)\}_{p\in E}$ is an open cover of $E$.  Now take a finite
subcover and recall that a finite union of finite sets is a finite.\qed
\end{prob}


\begin{prop}
If $E$ is a compact metric space, then every sequence $\la
p_n\ra_{n=1}^\infty$ has a convergent subsequence.
\end{prop}


\begin{prob}
Prove this. \hint  Split this into two cases.  {\sc Case 1.}  The set
$\{p_1,p_2,\ldots\}$ is infinite.  Then this set will have a cluster
point, $p$.  Show there is a subsequence of the sequence that
converges to $p$.  {\sc Case 2.}  The set $\{p_1,p_2,\ldots\}$ is
finite (an example of this would be the sequence of real numbers
$p_n=(-1)^n$ 
that only takes on two values).  Then there is 
a subsequence $\la p_{n_k}\ra_{k=1}^\infty$ which is constant, say
$p_{n_k}=p$ 
for all~$k$.\qed
\end{prob}


\begin{thm}
Let $E$ be a compact metric space and let $K_1,K_2,K_3,\ldots$ be a
sequnce of nonempty closed subsets of $E$ that are nested in the sense
that
$$
K_1\supseteq K_2\subseteq K_3\supset \cdots.
$$
Then 
$$
\bigcap_{n=1}^\infty K_n\ne \nothing.
$$
\end{thm}

\begin{prob}
Prove this. \hint Towards a contradiction assume $\bigcap_{n=1}^\infty
K_n= \nothing$, and show that if $V_n=\mathcal K_n$, then
$\{V_n\}_{n=1}^\infty$ is an open cover of $E$ and use that to get a
contradiction. \qed
\end{prob}


  \bibliographystyle{amsplain}
%\bibliography{/home/howard/texmf/inputs/HowRefs} %For use at school.
\bibliography{~/texmf/inputs/HowRefs} %For use at home.
 
\end{document}






















