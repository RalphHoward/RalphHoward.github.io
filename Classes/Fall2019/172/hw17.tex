
%%%  This document is a LaTeX2e document
%%%  It uses the AMS article style

\documentclass[11pt]{amsart}
\usepackage{enumerate}

%%% Make more symbols available
\usepackage{amsmath}
\usepackage{amssymb}
\usepackage{latexsym}
\usepackage{overpic}

%%% Some packages useful for graphics
% \usepackage{bezier}
\usepackage{epic,eepic}
%\usepackage{epsfig}            % For postscript
\usepackage{graphicx}

%%% Bold Italic font.  My convention is to use this font when defining
%%% terms. The usage is the same as \emph{}

\newcommand{\bi}[1]{{\bf\itshape #1\/}}  % Bold Italic (used for def'ing
                                  % terms)


\newcommand{\eq}[1]{~\eqref{#1}}

%%%
%%%  Sized delimiters.  Note \( and \) and \[ \] override
%%%  LaTeX's start and stop math mode
%%%

\renewcommand{\(}{\left(}
\renewcommand{\)}{\right)}
\renewcommand{\[}{\left[}
\renewcommand{\]}{\right]}

%%%  For Marginal Notes.  Useful during editing, especially with
%%%  several authors involved.

\catcode`@=11 \@mparswitchfalse  %This puts the \mnote's on the right.

\newcounter{mnotecount}[section]
\renewcommand{\themnotecount}{\thesection.\arabic{mnotecount}}
\newcommand{\mnote}[1]
{\protect{\stepcounter{mnotecount}}$^{\mbox{\footnotesize  $%\!\!\!\!\!\!\,
      \bullet$\themnotecount}}$ \marginpar{\null\hskip-9pt\vskip-30pt\raggedright\tiny\em
      \themnotecount:\! #1} }
%\newcommand{\mnote}[1]{}	 %Deletes mnotes for a final copy


\newcommand{\ralph}{{\tiny \bf *RH*} } % For labeling a \mnote as
                                        % being by Ralph Howard.
%assnewcommand{\co}{{\tiny\bf *CA*} }  % For labeling a \mnote as
                                        % being by Co Author.


%%%  Theorem type environments

% \newtheorem{theorem}{Theorem}  % I have used this for theorems in
                 % the introduction for labeling independent of the
                 % rest of the document.

%\swapnumbers  % Changes num Theorem to Theorem num.  Only effects the
               % environments listed below it.

\newtheorem{thm}{Theorem}
\newtheorem{lemma}[thm]{Lemma}
\newtheorem{prop}[thm]{Proposition}
\newtheorem{cor}[thm]{Corollary}
\newtheorem{claim}[thm]{Claim}

\theoremstyle{definition}

\newtheorem{notation}[thm]{Notation}
\newtheorem{defn}[thm]{Definition}
\newtheorem*{notdefn}{Negation of Definition of Continuity}

\theoremstyle{remark}

\newtheorem{remark}[thm]{Remark}
\newtheorem{example}[thm]{Example}
% \newtheorem{prob}{{\bf Problem}}


%%%%%% My standard abbreviations for mathematical symbols.  Some
%%%%%% have variants for compatibility with coauthors.

%%% Standard algebraic objects:


%%% Derivatives:

\newcommand{\f}{\partial}
%\newcommand{\p}[1]{{\frac{\partial}{\partial #1}}}  % d/d#1
                                                % (the vector field)
%\renewcommand{\d}{\,\mathsf{d}}   % Sometimes used as the "d" in an
                                  % integral.

%%% Geometric and topological objects:

\newcommand{\area}{\operatorname{Area}\nolimits}
\newcommand{\dist}{\operatorname{dist}}
\newcommand{\length}{\operatorname{Length}\nolimits}
\newcommand{\vol}{\operatorname{Vol}\nolimits}

\newcommand{\un}{{\mathbf n}}       % Unit Normal.
\newcommand{\ut}{{\mathbf t}}       % Unit Tangent.
\newcommand{\ub}{{\mathbf b}}       % Unit Binormal.
\newcommand{\ii}{\text{\sl I\!I}}   % Second fundamental form

%%% Misc.

\newcommand{\cd}{,\dots,}      % Commas with Dots. i.e. ,...,
\newcommand{\e}{\varepsilon}   % epsilon that looks small.
\newcommand{\ol}{\overline}   % OverLine
\newcommand{\cn}{\colon}
\newcommand{\la}{\langle}
\newcommand{\ra}{\rangle}

\newcommand{\ds}{\displaystyle}

\newcommand{\R}{{\mathbb R}}
\newcommand{\Z}{{\mathbb Z}}
\newcommand{\Q}{{\mathbb Q}}
\newcommand{\N}{{\mathbb N}}

\newcommand{\eqed}{\null\vskip-28pt\qed}

\newcommand{\hint}{\emph{Hint:} }

\newcommand{\qimplies}{\quad\text{implies}\quad}
\newcommand{\qtext}[1]{\quad\text{#1}\quad}
\newcommand{\comp}{{\mathcal C}}
%
% \newcounter{ProbCount}
%
% \newcommand{\prob}{\stepcounter{ProbCount}
% {\noindent{\bf \arabic{ProbCount}.} }}
%
% \newcounter{SubProbCount}[ProbCount]
% %\stepcounter{subpr}
% \newcommand{\sub}{\stepcounter{SubProbCount}(\alph{SubProbCount}) }

\newcommand{\an}{\emph{Answer:} }
\newcommand{\sol}{\smallskip \emph{Solution:  }}

\newcommand{\lhos}{l'h\^opital}
\newcommand{\Lhos}{L'h\^opital}


\newcounter{ProbCount}

\newcommand{\prob}{\stepcounter{ProbCount}
{\noindent{\bf \arabic{ProbCount}.} }}

\newcounter{SubProbCount}[ProbCount]
%\stepcounter{subpr}
\newcommand{\sub}{\stepcounter{SubProbCount}(\alph{SubProbCount}) }

% \newcommand{\an}{\emph{Answer:} }
% \newcommand{\sol}{\smallskip \emph{Solution:  }}




%%% Title

\title[]{}

%%% Subject and keyword information:

%\subjclass[2000]{}
%\keywords{}


\begin{document}
\thispagestyle{empty}

\centerline{\bf \large Mathematics 172 Homework.}



Recall the plot to date on our study of age structured population growth.
If we have three stages we let
$$
\vec n(t) = \[ \begin{matrix} n_1(t)& n_2(t) & n_3(t) \end{matrix}\]
$$
where $n_j(t)$ is the number of individuals in stage $j$ in year $t$.
We have the \bi{Leslie matrix} 
$$
L = \[ \begin{matrix}
f_1&f_2&  f_3\\
p_1 & 0& 0 \\
0& p_2& 0
\end{matrix}\].
$$
In entries have the meanings
\begin{align*}
f_j&=\text{Fecundity of stage $j$ individuals}\\
&= \text{average number of offspring to a stage $j$ individual.}\\ 
p_j& =\text{Proportion of stage $j$ individuals that live to stage $j+1$.}
\end{align*}
The Leslie matrix tells us how the population changes from one year to the next:
$$
\vec n(t+1) = L \vec n(t).
$$
Thus if we know $\vec n(0)$ we can compute the numbers in the future by
$$
\vec n(t) = L^t\vec n(0).
$$
At least for matrices that are not too large this is easy to do on our calculators.

One of the pieces of information that is \bi{age distribution} in year $t$.
That is the proportion of the population that is in each stage.
If
$$
\[ \begin{matrix} n_1(t)\\ n_2(t) \\ n_3(t) \end{matrix}\]
$$
then the total number of individuals is
$$
N(t) = n_1(t)+ n_2(t) + n_3(t) 
$$
and the age distribution is given by the vector
$$
\frac{1}{N(t)}\vec n(t)=
\frac{1}{N(t)}\[ \begin{matrix} n_1(t)\\ n_2(t) \\ n_3(t) \end{matrix}\]
=
\[ \begin{matrix} \dfrac{n_1(t)}{N(t)}\\
	\\ \dfrac{n_2(t)}{N(t)}\\ \\\dfrac{n_3(t)}{N(t)}\end{matrix}\]
$$
We computed many examples of this on Quiz 17.

We have a \bi{stable age distribution} if the age distribution stays the same from year to year.  
That this means is that for years $t$ and $t+1$ we have
$$
\frac{1}{N(t)}\vec n(t)=
\frac{1}{N(t+1)}\vec n(t+1)
$$
which can be rewritten as
$$
\vec n(t+1)= \(\frac{N(t+1)}{N(t)}\)\vec N(t)=\lambda \vec N(t)
$$
where
$$
\lambda = \frac{N(t+1)}{N(t)}.
$$
But we also have
$$
\vec  n(t+1) = L \vec n(t).
$$
Comparing our two formulas for $\vec n(t)$ gives
$$
L \vec N(t) = \lambda \vec N(t).
$$
In this case we let
\begin{align*}
	\lambda&= \text{\bi{growth ratio}}\\
	r = \lambda-1 & = \text{\bi{per capita growth rate.}}
\end{align*}
\bigskip

This motivates the following.
Let $L$ be a square matrix and $\vec n$ a vector and $\lambda$ a number.  Then $\vec n$ is
an \bi{eigenvector} of $L$ with \bi{eigenvalue} $\lambda$ if and only 
if 
$$
L \vec n = \lambda \vec n.
$$
What this means for us is that $L \vec n$ and $\vec n$ have the same age distribution.
Thus in the context of our class saying that $\vec n$ is a eigenvector of $L$ is
saying that $\vec n$ has the stable age distribution and that the eigenvector $\lambda$
is the growth ratio.



Let us see what this mean in concrete cases.  Let $L$ be the Leslie matrix
$$
L = \[ \begin{matrix}
0&44&  890\\
0.01 & 0& 0 \\
0& 0.8& 0
\end{matrix}\]
$$
and let
$$
\vec n = \[ \begin{matrix}
4\\ 0.02\\ 0.009
\end{matrix}\]
$$

\prob Show that
$$
L\vec n = \[ \begin{matrix}
8\\ 0.04\\ 0.016
\end{matrix}\]
$$
\qed

Also note that
$$
2 \vec n = 2  \[ \begin{matrix}
8\\ 0.04\\ 0.016
\end{matrix}\]
=   \[ \begin{matrix}
2(8)8\\2( 0.04)\\ 2(0.016)
\end{matrix}\]
=  \[ \begin{matrix}
8\\ 0.04\\ 0.016
\end{matrix}\].
$$
Thus $\vec n$ is a eigenvector for $L$ with eigenvalue $2$.
Therefore for this Leslie matrix the stable age distributions is
$$
\frac{1}{8+.02+.009}  \[ \begin{matrix}
8\\ 0.04\\ 0.016
\end{matrix}\]= 
\[ \begin{matrix}
0.99305\\
0.00497\\
0.00199
\end{matrix}
\],
$$
the growth factor is $\lambda = 2$ and $r = \lambda -1=1$ is the per capita 
growth rate.
\bigskip


\prob Let $L$ be the Leslie matrix
$$
L= \[ \begin{matrix}
0& 3.9& 15.25\\
0.1& 0&0\\
0&0.4& 0
\end{matrix}\]
$$
\sub Show that
$$
\vec n = \[ \begin{matrix}
100\\10\\4
\end{matrix}\]
$$
is an eigenvector for $L$ with eigenvalue $\lambda=1$ and thus that
per capita growth rate is $r = \lambda -1=0$.  

\sub Find the stable age distribution. \sol It is
$$
\[ \begin{matrix}
0.87719  \\ 0.08772  \\  0.03509
\end{matrix}\]
$$

\sub If
$$
\vec n(0) = \[ \begin{matrix}
100  \\ 0 \\  0
\end{matrix}\]
$$
find $\vec n(1)$, $\vec n(2)$, $\vec n(5)$ and $\vec n(30)$ and the 
age distribution of these vectors.

\sol 
$$
\vec n(1) = \[\begin{matrix} 0 \\10 \\ 0\end{matrix}\]\quad \text{age distribution is}\quad
\[\begin{matrix} 0 \\1.0 \\0 \end{matrix}\]
$$
$$
\vec n(2) = \[\begin{matrix} 39 \\0 \\ 4\end{matrix}\]\quad \text{age distribution is}\quad
\[\begin{matrix} 0.90698 \\ 0\\ 0.09302\end{matrix}\]
$$
$$
\vec n(5) = \[\begin{matrix} 47.5800 \\1.5210 \\ 2.4400\end{matrix}\]\quad \text{age distribution is}\quad
\[\begin{matrix} 0.92315 \\0.02951 \\0.04734 \end{matrix}\]
$$
$$
\vec n(30) = \[\begin{matrix} 38.3362 \\3.8266 \\ 1.5343\end{matrix}\]\quad \text{age distribution is}\quad
\[\begin{matrix} 0.87732 \\ 0.08757\\0.03511 \end{matrix}\]
$$











\end{document}

**********************************
 

*******  For loop diagrams
*******
\null\vskip.8in
\begin{figure}[htb]
\centering
\begin{overpic}[width=4.5in]{loops.pdf}
 \put(9,11){$1$}
 \put(49,11){$2$}
 \put(88,11){$3$}
 \put(28,13){.11} %%% $p_1$
 \put(67,13){.22} %%% $p_1$
 \put(21,20){\includegraphics[scale=.40]{f1.pdf}}
     \put(28,26){3.3}
 \put(59,20){\includegraphics[scale=.40]{f1.pdf}}
     \put(67,26){4.4}
 \put(14,24){\includegraphics[scale=.47]{f3.pdf}}
     \put(48,41){5.5}
 \put(2,-17){\includegraphics[scale=.5]{bot.pdf}}
 	 \put(9,-13){6.6}
 \put(40,-17){\includegraphics[scale=.5]{bot.pdf}}
 	 \put(48,-13){7.7}
 \put(78,-17){\includegraphics[scale=.5]{bot.pdf}}
 	 \put(86,-13){8.8}
\end{overpic}
% \caption[]{}
% \label{fig:name}
\end{figure}
\vskip.7in







 \includegraphics[scale=1]{graph.pdf}

 or

\begin{figure}[hb]
\includegraphics[scale=1]{graph.pdf}
\caption[]{}
\label{}
\end{figure}

parameters are    height  ,  width  ,   angle

**********************************



\begin{figure}[hb]
\centering
\input{graph1.eepic}
\caption[]{}
\label{}
\end{figure}




