\documentclass{article}  % Tell LaTeX what type of document we are making.
                         % Percent signs are used for comments and what is
                         % to the right of them is ignored.




\begin{document}  %  Start the document.


Hello world.   



Ordinary text (that is non mathematical text) is just typed
as usual ordinary text.  But LaTeX will decide where the line
breaks are.

A new paragraph is started by leaving one or more blank lines.

So this is a new paragraph.


% Mathematical formulas in a line are put between \(  and \)

The basic quadratic equation is \(ax^2 + bx + c =0\).  


% Displayed mathematical formulas are put between \[ and \]

This can be done in one line \[ ax^2+bx+c=0. \]
But it is usually easier to read your code if you make it look
displayed 
\[
  ax^2+bx+c=0.
\]
(LaTeX will not care, but if you are looking at your code a month after
writing it you will.)

Let us give a slightly more complicated displayed equation
which gives a good idea of what general LaTeX looks like:
\[
	x = \frac{-b \pm \sqrt{b^2-4ac}}{2a}.
\]

A basic formula in much of analysis is the sum of a geometric 
series
\[
   \frac{a}{1-r}= a+ar + ar^2 + ar^3 + ar^4+ \cdots = \sum_{n=0}^\infty ar^n
\]
which converges for $|r|<1$.
Thus if
\[
	S = 3 + 3 x + 3 x^2 + 2 x^3 + \cdots = \sum_{n=0}^\infty 3x^n
\]
we have
\[
      S =   \frac{3}{1-x} 
\]

\end{document}

