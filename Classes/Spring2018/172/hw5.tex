%%%  This document is a LaTeX2e document 
%%%  It uses the AMS article style  

\documentclass[11pt]{amsart}
 
%%% Make more symbols available 
\usepackage{amsmath}
\usepackage{amssymb}
\usepackage{latexsym}



%%% More flexible enumerate environments.

\usepackage{enumerate}


%%% Some packages useful for graphics 
% \usepackage{bezier}
\usepackage{epic,eepic}	
%\usepackage{epsfig}            % For postscript
\usepackage{graphicx}
 
%%% Bold Italic font.  My convention is to use this font when defining
%%% terms. The usage is the same as \emph{}

\newcommand{\bi}[1]{{\bf\itshape #1\/}}  % Bold Italic (used for def'ing 
                                  % terms) 
 

\newcommand{\eq}[1]{~\eqref{#1}}

%%%
%%%  Sized delimiters.  Note \( and \) and \[ \] override
%%%  LaTeX's start and stop math mode
%%%

\renewcommand{\(}{\left(}
\renewcommand{\)}{\right)}
\renewcommand{\[}{\left[}
\renewcommand{\]}{\right]}

%%%  For Marginal Notes.  Useful during editing, especially with 
%%%  several authors involved.

\catcode`@=11 \@mparswitchfalse  %This puts the \mnote's on the right.

\newcounter{mnotecount}[section]
\renewcommand{\themnotecount}{\thesection.\arabic{mnotecount}}
\newcommand{\mnote}[1]
{\protect{\stepcounter{mnotecount}}$^{\mbox{\footnotesize  $%\!\!\!\!\!\!\,
      \bullet$\themnotecount}}$ \marginpar{\null\hskip-9pt\vskip-30pt\raggedright\tiny\em
      \themnotecount:\! #1} }
%\newcommand{\mnote}[1]{}	 %Deletes mnotes for a final copy


\newcommand{\ralph}{{\tiny \bf *RH*} } % For labeling a \mnote as 
                                        % being by Ralph Howard.
%\newcommand{\co}{{\tiny\bf *CA*} }  % For labeling a \mnote as 
                                        % being by Co Author.


%%%  Theorem type environments

% \newtheorem{theorem}{Theorem}  % I have used this for theorems in 
                 % the introduction for labeling independent of the
                 % rest of the document. 

%\swapnumbers  % Changes num Theorem to Theorem num.  Only effects the   
               % environments listed below it.

\newtheorem{thm}{Theorem}
\newtheorem{lemma}[thm]{Lemma}
\newtheorem{prop}[thm]{Proposition}
\newtheorem{cor}[thm]{Corollary}
\newtheorem{claim}[thm]{Claim}
 
\theoremstyle{definition}

\newtheorem{notation}[thm]{Notation}
\newtheorem{defn}[thm]{Definition}

\theoremstyle{remark}

\newtheorem{remark}[thm]{Remark}
\newtheorem{example}[thm]{Example}
%\newtheorem{prob}{{\bf Problem}}


%%%%%% My standard abbreviations for mathematical symbols.  Some 
%%%%%% have variants for compatibility with coauthors. 

%%% Standard algebraic objects:


%%% Derivatives:

\newcommand{\f}{\partial}    
%\newcommand{\p}[1]{{\frac{\partial}{\partial #1}}}  % d/d#1 
                                                % (the vector field)
%\renewcommand{\d}{\,\mathsf{d}}   % Sometimes used as the "d" in an 
                                  % integral.

%%% Geometric and topological objects:

\newcommand{\area}{\operatorname{Area}\nolimits}
\newcommand{\dist}{\operatorname{dist}}  
\newcommand{\length}{\operatorname{Length}\nolimits}
\newcommand{\vol}{\operatorname{Vol}\nolimits}
 
\newcommand{\un}{{\mathbf n}}       % Unit Normal.
\newcommand{\ut}{{\mathbf t}}       % Unit Tangent.
\newcommand{\ub}{{\mathbf b}}       % Unit Binormal.
\newcommand{\ii}{\text{\sl I\!I}}   % Second fundamental form

%%% Misc.

\newcommand{\cd}{,\dots,}      % Commas with Dots. i.e. ,...,
\newcommand{\e}{\varepsilon}   % epsilon that looks small.
\newcommand{\ol}{\overline}   % OverLine
\newcommand{\cn}{\colon}
\newcommand{\la}{\langle}
\newcommand{\ra}{\rangle}

\newcommand{\hint}{\emph{Hint:} }



\newcounter{ProbCount}

\newcommand{\prob}{\stepcounter{ProbCount}
{\noindent{\bf \arabic{ProbCount}.} }}

\newcounter{SubProbCount}[ProbCount]
%\stepcounter{subpr}
\newcommand{\sub}{\stepcounter{SubProbCount}(\alph{SubProbCount}) }

\newcommand{\an}{\emph{Answer:} }
\newcommand{\sol}{\smallskip \emph{Solution:  }}



%%% Title

\title[]{}

%%% Subject and keyword information:

%\subjclass[2000]{}
%\keywords{}


\begin{document}
 

\thispagestyle{empty}


\centerline{\Large\bf Mathematics 172 Homework}\bigskip

We have seen that for an annual  organism that lives and reproduces once a year
and has an average number of $\lambda$ offspring that survive to the next year 
that if $P_t$ is the population size in year $t$ that
$$
P_{t+1} = \lambda P_t
$$
where $\lambda$ is the \bi{finite growth rate}.  This says that
to get the size of the population in year $t+1$ we just multiply
the population size in year $t$ by $\lambda $.

Let 
$$
P_0 = \text{Size of initial population.}
$$
Then we can get the size of the populations for the first several years:
\begin{align*}
P_1 &=\lambda P_0 = P_0 \lambda\\
P_2 &= \lambda P_1 = \lambda P_0\lambda = P_0 \lambda^2\\
P_3 &= \lambda P_2 = \lambda P_0\lambda^2 = P_0 \lambda^3\\
P_4 &= \lambda P_3 = \lambda P_0\lambda^3 = P_0 \lambda^4\\
P_5 &= \lambda P_4 = \lambda P_0\lambda^4 = P_0 \lambda^5\\
P_6 &= \lambda P_5 = \lambda P_0\lambda^5 = P_0 \lambda^6\\
P_7 &= \lambda P_6 = \lambda P_0\lambda^6 = P_0 \lambda^7\\ 
\end{align*}
At this point you see the pattern:
$$
P_t= P_0 \lambda^t.
$$

\begin{example}
One of the best known examples of an annual insects is the annual 
cicada which supply so much of the outdoors sound track
here is South Carolina in the late summer.
Assume that $20$ 
cicadas are introduced on an island and that the 
finite growth rate for them is $\lambda = 5.5$ cicadas/cicada.
Then what is a formula for the number of cicadas after $t$ years?
How many cicadas are there in five years?  How many long until there
are a million rats?  
\end{example}

\sol In this case we have $P_0= 20$ and 
$\lambda = 5.5$  Therefore the formula 
for the number in $t$ years is
$$
P_t = P_0 \lambda^t = 20 (5.5)^t.
$$
Therefore after ten years, that is $t=5$, the number of cicadas is
$$
P_{10}= 10(5.5)^{5} = 100{,}656.875\text{ cicadas}.
$$
(This can be rounded to the nearest cicada to get $P_5 \approx  100{,}657$ 
cicadas.
Finally to see how long until a million cicadas we want to solve:
$$
20(5.5)^t = 1{,}000{,}000 = 10^6,
$$
This gives
$$
(5.5)^t = (10^6)/20
$$
and therefore
$$
t \ln(5.5) = \ln\( 10^6/20\)
$$
and thus
$$
t = \ln\( 10^6/20\)/  \ln(5.5) = 6.3468487 \text{ years.}
$$

\prob 
In a backyard someone introduces $13$ annual weeds.  Let $P_t$
be the number of weeds $t$ year later.  Assume that 
the finite growth rate of the weeds is $3.3$ weeds/weed each year.

(a) Give a formula for the number of weeds after $t$ year.

(b)  How long until there are a million weeds in the yard?

(c) How many are there after a $52$ years.)

\sol (a) $P_t = 13 (3.3)^t$.  (b) Solve $13(3.3)^t = 10^6$ to get
$t = 9.423$ year,  (c) $P_{52} = 13(3.3)^{52} = 1.1931 \times 10^{28}$
weeds.

\begin{example}
Assume that $25$ sunflowers are introduced into a large field.  Sunflowers
are annuals.  Assume that after three years there are $40$ sunflowers in
the field.  Use this information to find a formula for the number of sunflowers
after $t$ years and use this to predict the number that will be in the field
after $10$ years.

\sol The number is $P_t=P_0\lambda^t$.  We know that $P_0=25$, but we 
still have to find $\lambda$.  We have
$$
P_3 = 25 \lambda^3 = 40.
$$
This leads to
$$
\lambda^3 = (40/25)
$$
and thus
$$
\lambda = (40/25)^{1/3} = 1.1696\ .
$$
Therefore
$$
P_t = 25 (1.1696)^t
$$
Therefore the number after $10$ years is
$$
P_{10} = 25*(1.1696)^{10} = 119.77 \text{ sunflowers}.
$$
\end{example}

\prob Chickweed (Stellaria media) is an annual plant that is considered a weed.
Assume that $9$ chickweeds are introduced into a large park and that $5$ 
years later there are $100$ chickweeds in the park.  

(a) Find a formula
for the both the number of chickweeds in the park after $t$ years. 

(b)  How many chickweeds are there in
the park after $10$ years?  

(c) How long until there are  $10{,}000$ chickweeds 
in the park?

\sol (a) First show that $\lambda =1.61864$ and 
therefore $P_t= 9(1.61864)^t$.  
(b) $P_{10} = 9(1.61864)^{10}= 1{,}111.1$ chickweeds.  
(c) $t=9.7812$ years.  
\bigskip


\centerline{\includegraphics[scale=.40]{../Images/Chickweed.jpg}}
\bigskip

\centerline{Photo of Chickweed.}
\bigskip






\end{document}

**********************************
 
 \includegraphics[scale=1]{graph.pdf}

 or

\begin{figure}[hb]
\includegraphics[scale=1]{graph.pdf}
\caption[]{}
\label{}
\end{figure}

parameters are    height  ,  width  ,   angle

**********************************



\begin{figure}[hb]
\centering
\input{graph1.eepic}
\caption[]{}
\label{}
\end{figure}




