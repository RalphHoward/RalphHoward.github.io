%%%  This document is a LaTeX2e document
%%%  It uses the AMS article style

\documentclass[11pt]{amsart}
\usepackage{enumerate}

%%% Make more symbols available
\usepackage{amsmath}
\usepackage{amssymb}
\usepackage{latexsym}
\usepackage{overpic}
\usepackage{marvosym}
\usepackage{hyperref}




%%% Some packages useful for graphics
% \usepackage{bezier}
\usepackage{epic,eepic}
%\usepackage{epsfig}            % For postscript
\usepackage{graphicx}

%%% Bold Italic font.  My convention is to use this font when defining
%%% terms. The usage is the same as \emph{}

\newcommand{\bi}[1]{{\bf\itshape #1\/}}  % Bold Italic (used for def'ing
                                  % terms)


\newcommand{\eq}[1]{~\eqref{#1}}

%%%
%%%  Sized delimiters.  Note \( and \) and \[ \] override
%%%  LaTeX's start and stop math mode
%%%

\renewcommand{\(}{\left(}
\renewcommand{\)}{\right)}
\renewcommand{\[}{\left[}
\renewcommand{\]}{\right]}

%%%  For Marginal Notes.  Useful during editing, especially with
%%%  several authors involved.

\catcode`@=11 \@mparswitchfalse  %This puts the \mnote's on the right.

\newcounter{mnotecount}[section]
\renewcommand{\themnotecount}{\thesection.\arabic{mnotecount}}
\newcommand{\mnote}[1]
{\protect{\stepcounter{mnotecount}}$^{\mbox{\footnotesize  $%\!\!\!\!\!\!\,
      \bullet$\themnotecount}}$ \marginpar{\null\hskip-9pt\vskip-30pt\raggedright\tiny\em
      \themnotecount:\! #1} }
%\newcommand{\mnote}[1]{}	 %Deletes mnotes for a final copy


\newcommand{\ralph}{{\tiny \bf *RH*} } % For labeling a \mnote as
                                        % being by Ralph Howard.
%\newcommand{\co}{{\tiny\bf *CA*} }  % For labeling a \mnote as
                                        % being by Co Author.


%%%  Theorem type environments

% \newtheorem{theorem}{Theorem}  % I have used this for theorems in
                 % the introduction for labeling independent of the
                 % rest of the document.

%\swapnumbers  % Changes num Theorem to Theorem num.  Only effects the
               % environments listed below it.

\newtheorem{thm}{Theorem}
\newtheorem{lemma}[thm]{Lemma}
\newtheorem{prop}[thm]{Proposition}
\newtheorem{cor}[thm]{Corollary}
\newtheorem{claim}[thm]{Claim}

\theoremstyle{definition}

\newtheorem{notation}[thm]{Notation}
\newtheorem{defn}[thm]{Definition}
\newtheorem*{notdefn}{Negation of Definition of Continuity}

\theoremstyle{remark}

\newtheorem{remark}[thm]{Remark}
\newtheorem{example}[thm]{Example}
\newtheorem{prob}{{\bf Problem}}


%%%%%% My standard abbreviations for mathematical symbols.  Some
%%%%%% have variants for compatibility with coauthors.

%%% Standard algebraic objects:


%%% Derivatives:

\newcommand{\f}{\partial}
%\newcommand{\p}[1]{{\frac{\partial}{\partial #1}}}  % d/d#1
                                                % (the vector field)
%\renewcommand{\d}{\,\mathsf{d}}   % Sometimes used as the "d" in an
                                  % integral.

%%% Geometric and topological objects:

\newcommand{\area}{\operatorname{Area}\nolimits}
\newcommand{\dist}{\operatorname{dist}}
\newcommand{\length}{\operatorname{Length}\nolimits}
\newcommand{\vol}{\operatorname{Vol}\nolimits}

\newcommand{\un}{{\mathbf n}}       % Unit Normal.
\newcommand{\ut}{{\mathbf t}}       % Unit Tangent.
\newcommand{\ub}{{\mathbf b}}       % Unit Binormal.
\newcommand{\ii}{\text{\sl I\!I}}   % Second fundamental form

%%% Misc.

\newcommand{\cd}{,\dots,}      % Commas with Dots. i.e. ,...,
\newcommand{\e}{\varepsilon}   % epsilon that looks small.
\newcommand{\ol}{\overline}   % OverLine
\newcommand{\cn}{\colon}
\newcommand{\la}{\langle}
\newcommand{\ra}{\rangle}

\newcommand{\ds}{\displaystyle}

\newcommand{\R}{{\mathbb R}}
\newcommand{\C}{{\mathbb C}}
\newcommand{\Z}{{\mathbb Z}}
\newcommand{\Q}{{\mathbb Q}}
\newcommand{\N}{{\mathbb N}}

\newcommand{\eqed}{\null\vskip-28pt\qed}

\newcommand{\hint}{\emph{Hint:} }

\newcommand{\qimplies}{\quad\text{implies}\quad}
\newcommand{\qtext}[1]{\quad\text{#1}\quad}
\newcommand{\comp}{{\mathcal C}}
%
% \newcounter{ProbCount}
%
% \newcommand{\prob}{\stepcounter{ProbCount}
% {\noindent{\bf \arabic{ProbCount}.} }}
%
% \newcounter{SubProbCount}[ProbCount]
% %\stepcounter{subpr}
% \newcommand{\sub}{\stepcounter{SubProbCount}(\alph{SubProbCount}) }

\newcommand{\an}{\emph{Answer:} }
\newcommand{\sol}{\smallskip \emph{Solution:  }}

\newcommand{\1}[1]{{\mathbf{#1}}}


%%% Title

\title[]{}

%%% Subject and keyword information:

%\subjclass[2000]{}
%\keywords{}


\begin{document}

\thispagestyle{empty}
\centerline{\Large\bf Mathematics 551 Take Home part of Test 1.}\bigskip

\noindent
\bi{This is due at the beginning of class on Wednesday, February~12.} 

\begin{prob} (20 points)
We have shown that if $\mathcal L(p,\theta)$ is the line
$$
\mathcal L(p,\theta) =\{ (x,y) : x\cos(\theta) + y \sin(\theta) = p\}
$$
then for any $C^1$ curve $\gamma$ that \bi{Crofton's Fromula}
$$
L(\gamma) = \frac{1}{2} \int_{0}^{2\pi}\int_{-\infty}^\infty \#(\gamma\cap \mathcal L(p,\theta))\,dp\,d\theta
$$
where $L(\gamma)$ is the length of $\gamma$.
\begin{enumerate}[\quad(a)]
\item Let $\gamma_1$ and $\gamma_2$ be convex curves with $\gamma_1$ surrounded 
by $\gamma_2$ as in this figure:

\centerline{
\begin{overpic}[width=2in]{home1pic1.pdf}
\put(19,30){$\gamma_1$}
\put(1,30){$\gamma_2$}
\end{overpic}
}

\noindent Prove $L(\gamma_1)< L(\gamma_2)$.

\item   Let $\gamma$ be a curve of length 1,000 that is contained inside
of the unit circle $x^2+y^2 = 1$:


\centerline{
\begin{overpic}[width=1.5in]{home1pic2.pdf}
\put(80,91){$x^2+y^2=1$}
\put(42,4){$\gamma$}
\end{overpic}
}

\noindent
Show there is a line that intersects $\gamma$ at least 160 times. \\
\hint
$1000/(2\pi)=159.1549\ldots$.\qed
\end{enumerate}
\end{prob}

\begin{prob} (20 points)
Let $\1c \cn [a,b] \to \R^2$ be a unit speed curve and assume the
curvature satisfies $\kappa(s)>0$.  Let $r>0$ and $\ut$ and $\un$
be the unit tangent and unit normal to $\1c$. Then the 
\bi{parallel curve} at distance $r$ from $\1c$ is the curve
$$
\1c_r = \1c(s) - r\un(s).  
$$

\begin{enumerate}[\quad(a)]
\item If the original curve $\1c$ is a circle of radius $R$
traversed so that the curvature
is positive, draw picture and explain why $\1c_r$ is the circle 
with radius $R+r$.  (This explains why it is natural to use
the negative in the definition of $\1c_r$.)
\item Show 
$$
\frac{d\1c_r}{ds} =\1c_r'(s) = (1+r\kappa(s))\ut(s)
$$
and use this to explain why the unit tangent and normal to $\1c_r$
are just $\ut(s)$ and $\un(s)$. Draw a picture to illustrate this.
\item Use Part (b) to show that length of $\1c_r$ is
$$
L(\1 c_r) = L(\1c) + r  \int_a^b \kappa(s)\,ds.
$$
\item Compute the curvature of $\1c_r$. \hint We know the
unit tangent and normal to $\1c_r$.  Let $\sigma$ be arclength
along $\1c_r$.  Then 
$$
\frac{d\sigma}{ds} = \| \1c_r'(s)\| = (1+r\kappa(s)).
$$
\eqed
\end{enumerate}
\end{prob}
\vskip.4in

\centerline{\large \bf Things you should know for the in class part of the 
test.}

\begin{enumerate}
\item The basic definitions and formulas for things such as 
length, velocity, speed, and curvature.

\item You definitely need to know the Frenet formulas.  

\item Be able to use that curvature is 
$$
\kappa = \frac{d\theta}{ds} 
$$
which implies that for a unit speed curve
$$
\int_a^b \kappa(s)\,ds = \int_a^b \frac{d\theta}{ds} \,ds = \theta(b) - \theta(a).
$$
This allows one to be able to compute integrals of curvature by
just seeing how much that tangent has rotated.  For example

\centerline{
\begin{overpic}[width=3.5in]{hom21pic3.pdf}
\put(,){}
\end{overpic}
}

\noindent 
In the first curve above $\theta$ changes form $\pi$ to $\pi/2$
and so for this curve $\int \kappa(s)\,ds = -\pi/2$.  For the second
curve the tangent goes through one resolution in the negative
direction and so for this curve $\int \kappa(s)\,ds = -2\pi$.

\item Know that statement of the isoperimetric inequality $4\pi A \le L^2$.

\item Know the statement of the maximum principle.  A reasonable 
question would be to outline a proof of the four vertex therm using
the maximum principle. 



\end{enumerate}

\end{document}

















**********************************

 \includegraphics[scale=1]{graph.pdf}

 or

\begin{figure}[hb]
\includegraphics[scale=1]{graph.pdf}
\caption[]{}
\label{}
\end{figure}

parameters are    height  ,  width  ,   angle

**********************************



\begin{figure}[hb]
\centering
\input{graph1.eepic}
\caption[]{}
\label{}
\end{figure}
